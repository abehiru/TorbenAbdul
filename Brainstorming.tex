%% LyX 2.2.3 created this file.  For more info, see http://www.lyx.org/.
%% Do not edit unless you really know what you are doing.
\documentclass[english]{article}
\usepackage[T1]{fontenc}
\usepackage[latin9]{inputenc}
\usepackage{babel}
\begin{document}
Suppose there is an economy that is trapped in a low income equilibrium.
In addition, suppose that this economy is located closer to equator
and the sea closer to Europe. This economy is endowed with a very
useful resource that has not been exploited yet {[}it is called sun
lights that can generate electricity 12 months particularly in winter
seasons of Europe{]}. Can trade in electricity with Europe lift this
country up from the trap or will it act as a natural resource curse?
What if there are multiple countries with this endowment? Do you see
a way to get more interesting ideas following this avenue? \bigskip{}

I think the question is interesting and the answer is non-trivial.
I assume that the bigger macro question you have in mind is: what
potential the sun resource has for unleashing economic development
in the continent and the globe? 

One important factor is the potential for resource curse. In this
regard, a crucial variable would be the amount of resource rent that
one can extract from \textquotedbl{}the sale of sunlight\textquotedbl{}.
By resource rent, I mean the portion of revenue from solar power (i.e.
power generated from sunlight, rather than the sunlight itself) that
can be captured by the \textquotedbl{}sunlight owner\textquotedbl{}.
In the case of oil, for example, if the oil price is 100 USD per barrel,
part of this price reflects extraction costs and the other part reflects
resource rents (e.g. in the form of royalties to governments in oil
producing countries). The question then is: how large can the size
of resource rent (royalties) be from the sale of solar power? This
should crucially depend on the supply of sunlight and the level of
competition among sunlight owners. If the supply is unlimited and
the competition among sunlight owners is very high, the resource rent
from solar power would be driven to zero.

{[}\textbf{Possible extensions}: what if countries form cartel/even
collude the production and rent distribution of the supply of power
by forming North Africa's Power Corporation, owned by these countries?
Since many of these countries also own oil, how does this play out
for transportation in European markets with substitution between electric
cars and oil using cars?{]}

\bigskip{}

The production technology is also important. The welfare effect of
owning sun-light should depend on the production technology. If the
production of solar power is labor-intensive, then some of the gain
might be captured by African countries in the form of employment income.
But if the production is capital-intensive, it may be very hard for
African countries to make income from export of solar power-{}-{}-the
competition among sunlight suppliers drives down resource rent and
taxing capital income could be very difficult due to the relatively
easy mobility of capital across national borders. Hence, most of the
income might be captured by capital owners (most likely international
investors). There is also a question of why do African countries care
about investment in solar power to begin with if the equilibrium outcome
is such that they end up with little benefit. So we may end up in
a situation where African countries do not provide sufficient incentive
for investment in solar power, and, as a result, we may end up with
inefficiently low level of investment in solar power. So the lack
of resource rent may lead to a different type of \textquotedbl{}resource-curse\textquotedbl{}.
{[}\textbf{Note}: It is capital intensive. The panels that capture
the sun light and convert it to electricity are like printout from
your printing machine. The printing machine is often produced in Germany,
the panels are ``printed out'' in mass in China. The choice of whethere
to import the printer or the printouts for African countries is not
clear. How does pricing of capital goods foster or hinder the emergence
of this technology?{]}

\bigskip{}

Another relevant question is the cost of transporting electricity
through grids from Africa to nearby continents (including Europe).
This is something you have the expertise. Let's consider two extreme
scenarios: the cost of transport is (1) infinity and (2) zero. What
is the implication of the transport cost for resource rents, resource
curse and location of production? For example, if countries can't
export the electricity, firms can relocate their production activities.
That is, when the cost of transporting electricity through grids is
prohibitively expensive, the production of energy-intensive products
should relocate to energy-rich locations. This may counter the forces
driving resource curse since, with low level of resource rent, political
entrepreneurs may have limited room to get rich through resource rents.
Instead, in order to attract investment in energy-intensive products,
they will rather have the incentive to make the environment more conducive
for investment. In that sense, solar power could be a resource blessing. 

\bigskip{}

I see a great potential in this question. We need to decide on our
focal question. One possibility could be the following model.

Consider the production of solar power in a world with two regions:
north and south (or a continuum of countries on the north-south line,
i.e. \textquotedbl{}distance from the sun\textquotedbl{}). The south
is rich in sunlight. The north is rich in capital. We may ask 1. How
efficient is the equilibrium level of power production? How does the
efficiency depend on the cost of transporting solar power and production
technology? 2. Which region benefits more/less? 3. What are the political
economy implications (here we can consider a short-cut assumption
that resource rents create perverse effect)? 4. What are the implications
for taxes/subsidies on/to the production of solar power?
\end{document}
